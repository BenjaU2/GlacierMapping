\chapter*{Introducción}
\addcontentsline{toc}{chapter}{Introducción}
\doublespacing
Esta investigación analiza el retroceso del glaciar Quelccaya mediante un estudio temporal desde 1991 hasta 2024. Para estimar la superficie glaciar a lo largo del tiempo, se emplearon imágenes multiespectrales de la misión Landsat y técnicas de deep learning. El objetivo principal de este proyecto es aplicar técnicas de deep learning para clasificar y segmentar cuerpos glaciares a partir de imágenes satelitales, permitiendo así realizar un análisis temporal y estimar el retroceso de la superficie del glaciar Quelccaya, ubicado en la Cordillera de Vilcanota, Perú. El estudio está estructurado en seis capítulos que abordan diversos aspectos del proyecto.

En el Capítulo 1, se presentan los aspectos generales de la investigación, incluyendo el área de estudio y la importancia de monitorear este glaciar, que tiene relevancia como fuente de agua dulce, además de su valor económico y cultural. También se definen los objetivos, hipótesis, alcance del estudio.

El Capítulo 2 proporciona el marco teórico necesario para comprender los conceptos básicos que sustentan los objetivos de este estudio, explorando temas relacionados con teledetección, deep learning y métodos de procesamiento de imágenes.

El Capítulo 3 se centra en el diseño de un sistema de segmentación de cuerpos glaciares. Aquí se detalla la metodología, desde la creación de la base de datos, que incluye adquisición y acondicionamiento de datos, generación de máscaras y tratamiento de imágenes, hasta la implementación de arquitecturas de deep learning. Se presentan los resultados de cada arquitectura evaluada para determinar la más adecuada para esta tarea.

En el Capítulo 4, se describen los experimentos y resultados obtenidos. Se detallan los experimentos realizados con las distintas arquitecturas propuestas, identificando la más óptima para la segmentación de cuerpos glaciares. Se analiza el retroceso de la superficie glaciar entre 1991 y 2024 utilizando las imágenes segmentadas por la arquitectura seleccionada, y se realiza un análisis temporal de la superficie glaciar en los años estudiados.

En Capítulo 5 se comparan los resultados obtenidos con aquellos de estudios previos que también investigaron el retroceso del glaciar Quelccaya. Además se compara la métodologia de esta investigación con la métodologia de los estudios previos.

Finalmente, en el Capítulo 6 se realizó la discusión, donde se interpretan, analizan y comparan los resultados obtenidos.

%En el Capítulo 7, se realiza las conclusiones del estudio, enfocandose en las observaciones más importantes.

%Finalmente en el Capitulo 8, se mencionan las recomendaciones, observaciones y trabajos a futuro que se pueden conciderar para mejorar los resultados del estudio.
\singlespacing