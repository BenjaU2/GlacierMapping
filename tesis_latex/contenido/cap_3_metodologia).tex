\chapter{Metodología}
\doublespacing
\begin{comment}
	contenidos...

El glaciat Quelccaya (coordenadas), se encuentra sobre una amplia meseta que se encuentra en la Cordillera Vilcanota del sur del Perú, entre los departamentos de CUsco y Puno. Es considerada la segunda capa de hilo tropical mas grande () La elevación media del margen de hielo es de 5300 m sobre el nivel del mar (m a.s.l.) y la elevación aproximada de la cumbre es de 5680 m a.s.l, por tanto el glaciart Quelccaya es representativa de muchos glaciares tropicales en los Andes con una elevación de la cumbre realtivamente baja . Las elevaciones más bajas alcanzadas por los glaciares más grandes de los Andes tropicales son tipicamente cercanas a 4850-4900 m.s.n.m., mietras que sus alcances supoeriores son frecuentemente superiores a 6000 m.s.n.m. 

\section{Características principales}
\subsection{Ubicación geográfica}
Ubicada en las montañas de la región de Cusco y Puno (), al sureste de Perú, cerca de la laguna Sibinacocha, una de las lagunas más importantes de la zona (Thompsom).
\subsection{Tamaño y extención}
En estudios realizados a principios del siglo xx1, se estima que el glaciar cubría un área aproximada de 44 km2, aunque este tamaño a disminuido considerablemente debido al calentamiento global (thompson), según taylor en el año 2020 reporto una estimación glaciar de 41.58 km2. Además, el espesor del glaciar varía, alcanzando hasta 100 metros en algunas de sus áreas mas altas (thompson) (schaefer). Sin embargo, investigaciones más recientes sugieren que tanto el tamaño como el espesor del glaciar se han reducido significativamente, con una pérdida estimada del 30 al 50 \% de su masa glaciar desde las últimas décadas del siglo XX (Schaefer et al., 2015).

\subsection{Glaciar tropical}

Quelccaya es un glaciar que se encuentra en una región tropical, lo que lo convierte en un ejemplo excepcional de glaciar en un clima cálido. Mientras que los glaciares suelen encontrarse en zonas polares o de altas latitudes, Quelccaya es un glaciar tropical que funciona como un importante indicador del cambio climático en zonas altas tropicales.
\subsection{Importancia científica}
\subsubsection{Estudio del cambio climatico}

El Quelccaya es una fuente clave de información para los científicos que estudian el cambio climático. Al ser un glaciar tropical, su comportamiento responde de manera más directa a los cambios en la temperatura y las precipitaciones. Esto porque los glaciares tropicales son más sensibles a las variaciones en la temperatura y las precipitaciones debido a que están en zonas donde las temperaturas son relativamente más altas.

Los investigadores han utilizado núcleos de hielo extraídos del glaciar para analizar su composición química y las capas de nieve. Estos nucleos son bibliotecas congeladas, la nieve que se acumula en los glaciares cada año capta concentraciones atmosfericas de polvo, sales marinas, cenizas, burbujas de gas y contaminación humana. Además el análisis de las propiedades físicas y quimicas puede revelar variaciones climaticas desde hace millones de años atraz y puede utilizarse para reconstruir la temperatura, la fuerza de circulación atmosferica, las precipitaciones, volumen de osceanos, erupción volcanica, varabilidad solar, extención de desiertos e incendios forestales, entre otros.

\subsubsection{Impacto del retroceso glaciar}

Como muchos otros glaciares en el mundo, el Quelccaya ha estado experimentando un retroceso acelerado en las últimas décadas. Este fenómeno está relacionado con el aumento de las temperaturas globales, y su desaparición parcial o total tendría implicaciones para el recurso hídrico de la región, según \parencite{worldweathernetwork2024} este glaciar podria desaparecer a finales del siglo \romannum{21}

\subsubsection{Investigaciones relevantes}
Uno de los estudios más notables sobre el Quelccaya fue llevado a cabo por un equipo de científicos liderado por el glaciólogo Lonnie Thompson, de la Universidad Estatal de Ohio, quien ha estado analizando los núcleos de hielo extraídos del glaciar desde la década de 1970. Estos estudios han permitido obtener información valiosa sobre el clima prehistórico, la variabilidad climática y los efectos del calentamiento global
\end{comment}
 
\section{Tipo de investigación}
El presente estudio se enmarca dentro del ámbito de la investigación científica e investigación aplicada, específicamente en el campo de ingeniería, puesto que se pretende aplicar técnicas de aprendizaje profundo, para abordar un problema específico y práctico con un enfoque en el análisis temporal del retroceso superficial del glaciar Quelccaya y su gran impacto en el medio ambiente.  
\section{Nivel de investigación}
La presente investigación es un estudio científico debido a su nivel de características.
\section{Alcance de la investigación}
Los alcances de este estudio pueden resumirse en:
	\begin{itemize}
		%\item Desarrollo de un sistema de segmentación y delimitación de cuerpos glaciares, este sistema será capaz de medir la superficie del cuerpo glaciar atrevés de algoritmos inteligentes y procesamiento de imágenes.
		%\item Implementación de una conjunto de datos de cuerpos glaciares de todo el Perú, con el objetivo de entrenar y validar el sistema de segmentación de cuerpos glaciares, dicho conjunto de datos tiene que ser realizado con los glaciares existentes en Perú.
		%\item Búsqueda y evaluación de modelos de CNN para segmentar y delimitar cuerpos glaciares. Estos modelos serán entrenados y evaluados para posteriormente compararlos para determinar cuál de ellos es el más adecuado para segmentar cuerpos glaciares.
		\item Desarrollar, entrenar y evaluar modelos de deep learning específicos para la clasificación y segmentación de cuerpos glaciares utilizando imágenes satelitales multiespectrales.
		\item Recopilar y etiquetar una base de datos de almenos mil imágenes obtenidos de imágenes satelitales que contengan cuerpos glaciares, asegurando su accesibilidad para el entrenamiento, validación y testeo de los modelos de deep learning.
		\item Realizar un análisis temporal del retroceso de la superficie glaciar utilizando los modelos de deep learning entrenados para identificar cambios en la superficie glaciar a lo largo del tiempo.
		\item Comparar los resultados obtenidos de la estimación de superficie glaciar con datos de referencia provenientes de artículos científicos, evaluando la precisión y fiabilidad del modelo.
	
	\end{itemize}
\section{Población}
Glaciar Quelccaya.
 
\section{Muestra}
En el presente estudio, la muestra se compondrá de imágenes multiespectrales con contengan cuerpos glaciares de diferentes zonas del Perú y parte de Bolivia para la creación de la base de datos, y se centrará específicamente en el glaciar Quelccaya para el análisis detallado del retroceso de superficie glaciar.
\section{Unidad de análisis}
\begin{itemize}
	\item Imágenes satelitales de los cuerpos glaciares.
	\item Superficie glaciar
\end{itemize}

\section{Técnicas e intrumentos de recolección de datos}
%En este estudio se utilizara imagenes multiespectrales obtenidas por satelites Sentinel o Landsat, la descarga de imágenes es a través de las plataformas USGS EARTH EXPLORER y ESA Copernicus Open Access.

En el presente estudio, se utilizarán imágenes satelitales multiespectrales que contien cuerpos glaciares de diferentes zonas del Perú, incluyendo el glaciar Quelccaya. Estas imágenes serán procesadas utilizando procesamiento de imágenes. Además, se etiquetara los cuerpos glaciares en las imágenes, creando así una base de datos etiquetada para el entrenamiento de modelos de deep learning.

\section{Plan de análisis de datos}
\begin{itemize}
	\item Los datos recopilados para este trabajo consisten en imágenes satelitales multiespectrales obtenidas por satélites, la descarga de imágenes es a través de las plataformas de libre acceso, así como sistema inteligente que hará la predicción de segmentación de cuerpos glaciares para posteriormente evaluar y analizar el retroceso de superficie glaciar.
	\item Preprocesamiento de datos. Se extraerán las características más importantes y se reducirán las dimensiones de las imágenes para simplificar el modelo.
	\item Métodos estadísticos y análisis. La precisión del modelo se medirá haciendo uso de técnicas mean intersection over union (mIOU), dice coefficient(Dice) y pixel accuracy (PA).
	\item Software. Se usará un lenguaje de alto nivel Python, junto con librerías como OpenCV, rasterio, GEDAL, Pytorch, Tensorflow, Pillow, entre otras.
\end{itemize}
%############################


\begin{comment}
	\section{Viabilidad y factibilidad}
	\begin{itemize}
		\item Disponibilidad de datos: Se ofrecen una gran cantidad de datos de imágenes satelitales de forma gratuita a través de plataformas de libre acceso, como el Servicio Geológico de los Estados Unidos (USGS). Esto hace que sea relativamente fácil acceder a las imágenes necesarias para tu investigación.
		
		\item Metodología establecida: La segmentación semántica utilizando deep learning se ha utilizado con éxito en una variedad de campos, incluida la detección de cambios en la cobertura terrestre. Hay varias arquitecturas de redes neuronales pre-entrenadas disponibles que se puede adaptar para el análisis de imágenes de glaciares.
		
		\item Avances en tecnología informática: Con el aumento de la capacidad de procesamiento y el acceso a herramientas de deep learning, realizar análisis de imágenes satelitales a gran escala se ha vuelto más factible. Los marcos de trabajo de deep learning como TensorFlow o PyTorch proporcionan una infraestructura robusta para desarrollar y entrenar modelos de segmentación semántica.
		
		\item Relevancia e importancia del tema: El retroceso de la superficie de los glaciares es un problema ambiental importante y de interés mundial debido a su relación con el cambio climático. Está investigación tiene el potencial de contribuir al entendimiento de este fenómeno y sus implicaciones.
		
		\item Costo relativamente bajo: En comparación con estudios de campo o la adquisición de datos de imágenes satelitales de alta resolución, el costo de está investigación puede ser relativamente bajo, especialmente si se aprovechan los recursos gratuitos disponibles.
		
	\end{itemize}
	
	\section{Limitaciones de la investigación}
	Algunas limitaciones de este estudio pueden resumirse en:
	\begin{itemize}
		
		%\item Precisión de las imágenes de satélite: Aunque Landsat proporciona imágenes de alta resolución, la calidad puede variar debido a factores como la cobertura de nubes, la distorsión atmosférica y la calidad de la imagen en sí misma. Estos factores pueden afectar la precisión del análisis.
		
		%\item Abundancia de sombras: Las sombras pueden distorsionar la información sobre el retroceso glaciar, especialmente en áreas montañosas donde los glaciares tienden a tener un relieve abrupto. El análisis podría subestimar o sobreestimar el retroceso glaciar.
		
		%\item Disponibilidad de datos históricos: Posibles dificultades para obtener imágenes desde 1990 hasta 2023 de alta calidad y sin nubes para todos los años y todas las estaciones. La disponibilidad de datos históricos puede variar según la ubicación y la época del año.
		
		%\item Validación de resultados: La segmentación semántica y el análisis de imágenes de manera automatizada pueden producir resultados precisos, pero es importante validar esos resultados con datos de campo o imágenes de alta resolución para garantizar su precisión.
		
		%\item Escalabilidad del método: La segmentación semántica mediante deep learning puede ser computacionalmente intensiva y requerir grandes cantidades de datos de entrenamiento.
		
		\item La capacidad del modelo para generalizar correctamente a nuevos datos y diferentes regiones geográficas puede ser limitada, afectando su aplicabilidad en diferentes contextos.
		\item La falta de datos satelitales históricos y etiquetados de alta calidad puede afectar la precisión y el entrenamiento del modelo de deep learning.
		\item La variabilidad estacional y los cambios climáticos pueden dificultar la clasificación y segmentación precisa de los cuerpos glaciares, ya que su apariencia puede variar significativamente a lo largo del año.
		\item Las discrepancias entre los resultados del modelo y los datos de referencia de artículos científicos pueden surgir debido a diferencias en los métodos de recolección de datos y análisis.
		
	\end{itemize}
	contenidos...
\end{comment}




\singlespacing