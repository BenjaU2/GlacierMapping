\chapter*{Discusión}
\doublespacing
Este estudio empleó un modelo de deep learning U-Net para segmentar los cuerpos glaciares del segundo glaciar más grande del mundo, el Quelccaya en Perú, utilizando imágenes de teledetección de Landsat correspondientes a varios años. El objetivo fue analizar los cambios ocurridos en el glaciar a lo largo de las últimas tres décadas.
\section{Análisis del modelo}
En la sección de resultados de entrenamiento, los modelos propuestos fueron evaluados durante 25 épocas, ya que se observó que no se obtenían mejoras significativas después de este número de iteraciones. Para lograr un rendimiento óptimo, fue necesario ajustar los valores de hiperparámetros, los cuales se presentan en la Tabla \ref{tabla2}. Este ajuste se realizó mediante múltiples entrenamientos para identificar la configuración más adecuada. Las Figuras \ref{fig:Miou_vs_epochs}, \ref{fig:PA_vs_epochs} y \ref{fig:DICE_vs_epochs} muestran los resultados de las métricas MIoU, PA y Dice, evaluadas para cada modelo. En estos gráficos, U-Net y DeepResUnet presentan un comportamiento consistente y similar, lo que sugiere que ambos modelos logran una segmentación semántica efectiva. En contraste, el modelo DeeplabV3Plus no alcanzó un desempeño competitivo en comparación con U-Net y DeepResUnet.

\section{Análisis del rendimiento de segmentación de glaciares}

La Figura \ref{fig:Loss_vs_epochs} muestra la evolución de la función de pérdida en cada modelo durante las evaluaciones. Aquí, U-Net y DeepResUnet logran una convergencia más efectiva, mientras que DeepLabV3Plus mantiene una pérdida elevada en comparación, indicando un menor ajuste en la segmentación.

Los resultados cuantitativos están resumidos en la Tabla \ref{tabla3}, donde se observa que DeepLabV3Plus tiene el menor rendimiento en segmentación según las métricas. DeepResUnet mejora considerablemente sobre DeepLabV3Plus al evaluar las mismas imágenes y métricas, mientras que U-Net destaca como el modelo de mayor precisión en la segmentación de cuerpos glaciares, alcanzando una precisión MIoU del 98.10\% y una pérdida de entrenamiento del 1\%.

Comparando con el estudio de \parencite{rajat2022glacier}, que también mapea superficies glaciares usando imágenes satelitales, U-Net mostró inicialmente una precisión de píxeles baja en su caso, aumentando rápidamente después de la época 20. En su estudio, entrenaron U-Net durante 200 épocas, obteniendo una precisión del 95.89\% y una tasa de pérdida de entrenamiento de solo 0.07\%. Este menor valor de pérdida podría deberse a la mayor cantidad de épocas (200) en comparación con las 25 épocas utilizadas en este estudio, así como a posibles diferencias en la cantidad y calidad de datos de entrenamiento.

En cuanto a la precisión observada en este estudio, U-Net muestra un rendimiento superior en comparación con \parencite{rajat2022glacier}. Esto puede estar relacionado con el uso de 2,400 imágenes del conjunto de datos, de las cuales el 70\% se destinó a entrenamiento, y  además se iplementó técnicas de aumento de datos en el algoritmo de entrenamiento.

Basado en la integración de las bandas 2, 3, 4, 5, 6 y 7 para Landsat 8 y las bandas 1, 2, 3, 4, 5 y 7 para Landsat 5 Collection 2 Level-1, U-Net demostró ser el modelo de mejor rendimiento, con una alta precisión en la clasificación y segmentación de diversos glaciares.

DeepResUnet también mostró un comportamiento similar a U-Net en cuanto a precisión; sin embargo, tiene el mayor número de parámetros de entrenamiento en comparación con U-Net y DeepLabV3Plus, debido al uso del mecanismo residual que, aunque mejora el rendimiento, también incrementa el costo computacional.

\section{Segmentación de glaciares y variación de glaciares}

Como se observa en la Figura \ref{fig:comparative_predictions}, los modelos propuestos presentan diferentes niveles de precisión en la segmentación de cuerpos glaciares. DeepLabV3Plus, aunque aceptable en términos generales, no logró segmentar los detalles finos de los cuerpos glaciares con la precisión esperada. En contraste, DeepResUnet mostró una segmentación más detallada de los límites glaciares, superando a DeepLabV3Plus en su capacidad para capturar los contornos de manera más completa. Sin embargo, el mejor rendimiento fue alcanzado por U-Net, que proporcionó la segmentación más precisa y detallada, demostrando su eficacia en la delineación exacta de los bordes glaciares.

En la Figura \ref{fig:imagen_temporal}, se presentan las imágenes binarias que reconstruyen la segmentación del glaciar Quelccaya desde 1991 hasta 2024 utilizando el modelo U-Net. Claramente, la extensión glaciar disminuye a lo largo de los años, y las áreas señaladas en los recuadros rojos de la Figura \ref{fig:comparacion} ilustran glaciares que desaparecen progresivamente. Sin embargo, la estimación precisa de la extensión glaciar a partir de imágenes satelitales requiere evitar escenas con nieve efímera, ya que esta puede introducir errores significativos en la estimación de la superficie glaciar. 

En la Figura \ref{fig:predicciones_temporales} Se observa claramente un retroceso significativo en la superficie glaciar a lo largo del tiempo, siendo más pronunciado en el lado occidental del glaciar en comparación con el lado oriental. Además, conforme el área glaciar ha disminuido, según los estudios de \parencite{inaigem2023} se ha evidenciado la formación y expansión de varias lagunas en la zona.

A diferencia de \parencite{malone2022evolution}, \parencite{taylor2022multi} y \parencite{hanshaw2014glacial}, quienes utilizaron el metodo NDSI para segmentar y extraer cuerpos glaciares, en este estudio se utilizó modelos de aprendizaje profundo para realizar la misma tarea, calcular los errores de las estimaciones no es una practica común, a pesar de la importancia de estimar correctamente la duperficie glaciar, segun \parencite{montoya2024estimation} si existe una clasificación errónea de los píxeles se debe a la insertidumbre del umbral NDSI, para ello es comun trabajar con umbrales que oscilen entre 0.55 y 0.66.

Es importante notar que el ritmo de disminución de la superficie glaciar varía anualmente. La tecnología satelital, combinada con herramientas avanzadas de segmentación automática, facilita el monitoreo de glaciares en retroceso, aunque sigue enfrentando desafíos, especialmente en la diferenciación entre agua, sombras y campos de nieve. Esto es particularmente relevante para glaciares cubiertos parcial o totalmente por escombros, donde una mayor resolución espacial ofrece ventajas significativas, no solo en términos de precisión de los límites glaciares, sino también en la visibilidad de características específicas, como grietas y texturas superficiales. Cabe mencionar que se estan realizando muchos esfuerzos para estudiar la deglaciación en los Andes peruanos. Sin embargo estos estudios son complejos debido a los numerosos picos montañosos y a las caracteristicas únicas en cuanto a ubicación e interacción con el medio ambiente. 

En general, la estimación de la superficie glaciar de Quelccaya obtenida en este estudio es varia ligeramente en comparación con estudios previos. Esto se debe en parte a que el modelo U-Net mejora la propagación de características, reduce la pérdida de información, aumenta el peso de las secciones de glaciares y maneja la amplia gama de categorías de píxeles del glaciar. 

Además fue optimizado para segmentar específicamente las superficies glaciares, lo que aumenta la confiabilidad de los resultados obtenidos. Este método automático permite segmentar el glaciar en cada imagen multiespectral de forma precisa, lo que respalda la viabilidad de esta técnica para estudios futuros de monitoreo glaciar. Estas imágenes multiepectrales de teledetección son muy buenas para segmentar cuerpos glaciares y su convinación con aprendizaje automatico tienen sus propias ventajas sobre el método tradicional NDSI de extracción de glaciares.

\singlespacing

\chapter*{Conclusiones}
\doublespacing
\begin{enumerate}
	\item Se aplicaron con éxito técnicas de deep learning para clasificar y segmentar cuerpos glaciares a nivel de píxel en imágenes multiespectrales. Los modelos utilizados mostraron un rendimiento eficaz en la clasificación y segmentación de estos cuerpos, logrando resultados precisos. Sin embargo, fue necesario seleccionar el modelo de aprendizaje profundo con mejor desempeño, y se eligió U-Net por los resultados sobresalientes que presentó.
	
	El análisis se realizó con imágenes del glaciar Quelccaya tomadas entre 1991 y noviembre de 2024. Para asegurar la precisión, se seleccionaron manualmente imágenes libres de nieve temporal y nubes, evitando así resultados falsos. Hubo algunos años en los que no se encontraron imágenes adecuadas, pero se incluyeron todas las disponibles para analizar y evaluar la estimación de la superficie glaciar de Quelccaya.
	
	A partir de este conjunto de datos, se efectuó un análisis temporal que muestra una clara disminución en la extensión del glaciar a lo largo de las décadas. Sin embargo, se observaron algunos años en los que la superficie glaciar aumentó ligeramente, probablemente debido a la presencia mínima de nieve temporal o a variaciones meteorológicas. Este patrón de incremento también aparece en otros estudios con los que se compararon los resultados.
	
	Finalmente, el método de segmentación mediante modelos de deep learning y el análisis temporal del retroceso de la superficie glaciar fueron llevados a cabo satisfactoriamente.
	
	\item Se utilizaron modelos basados en redes neuronales convolucionales (CNN) para la segmentación semántica de cuerpos glaciares, empleando tres arquitecturas: U-Net, DeepResUNet y DeepLabV3Plus. Los tres modelos demostraron ser efectivos en la tarea de segmentación, gracias a sus arquitecturas encoder-decoder, que permiten aprender características relevantes de los glaciares de manera automática. En comparación con los métodos tradicionales como el índice NDSI, que a menudo es lento y generalmente brinda resultados inapropiados al calcular incorrectamente el agua como nieve, los modelos de deep learning presentaron mejores resultados, con U-Net destacándose por su alta precisión. Las métricas obtenidas con U-Net fueron sobresalientes, con un MIoU de 0.9810, PA de 0.9979, Dice Coefficient de 0.9904 y un Loss de 0.01. Esto valida la hipótesis de que los algoritmos de deep learning pueden ofrecer una precisión significativa en la segmentación de cuerpos glaciares. Además, se sugiere que incluir más variables relevantes, como factores climáticos o topográficos, podría mejorar aún más el rendimiento del modelo.
	
	\item Se logró desarrollar una base de datos confiable de 2400 imágenes satelitales, con una distribución de datos del 70\% para entrenamiento, 15\% para validación y 15\% para testeo. Para garantizar la calidad de los datos, se seleccionaron cuidadosamente imágenes multiespectrales que contenían exclusivamente cuerpos glaciares, y se etiquetaron solo los píxeles correspondientes a características glaciares, estas etiquetas se pueden observar en la Figura \ref{fig:rgb_mascara}. Además, se utilizó Google Earth para verificar que los píxeles etiquetados realmente correspondieran a áreas glaciales. Con esta base de datos, los modelos de deep learning lograron una segmentación muy buena y efectiva de los cuerpos glaciares, tal como se puede observar en la Figura \ref{fig:comparative_predictions}, donde los tres modelos segmentan cuerpos glaciares con una alta precisión para clasificar píxeles con características glaciares, validando la hipótesis de que la creación de una base de datos adecuada contribuye al rendimiento de los modelos, aunque la magnitud de la precisión puede variar.
	
	\item Se analizaron los cambios en la superficie glaciar utilizando las imágenes segmentadas por el modelo U-Net y se cuantificó el área glaciar en diferentes años mediante un análisis temporal. Los resultados muestran que la extensión glaciar en 1991 era de 50.8131 km², mientras que en noviembre de 2024 se redujo a 35.4951 km², lo que representa una tasa de retroceso de aproximadamente 0.3936 km² por año. Sin embargo, en algunos años, las estimaciones mostraron un área mayor que la de años previos, lo que puede estar relacionado con la presencia de nieve temporal en las imágenes. Como se puede evidenciar en el año 2010, la estimación de área glaciar para este estudio fue de 40.0995 km² y para el año 2014 la estimación glaciar fue de 43.3746 km²; sin embargo, tal fenómeno también ocurrió con el análisis de \parencite{malone2022evolution}, quien en el año 2010 obtuvo 41.91 km2 y para el año 2014 obtuvo una estimación glaciar de 42.37 km².
	
	Para obtener estimaciones más precisas, es esencial evitar las imágenes que contengan nieve temporal, ya que esto puede introducir errores significativos en la medición. Se recomienda analizar múltiples imágenes durante la estación seca (junio-septiembre) y seleccionar aquellas con mínima presencia de nieve temporal. No obstante, incluso con estas precauciones, la estimación de la superficie glaciar puede seguir siendo incierta debido a las variaciones climáticas y las condiciones de cobertura en cada año. Además, algunos años no fueron considerados debido a la presencia excesiva de nieve temporal, nubosidad alta o la falta de imágenes disponibles en las fechas correspondientes.
	
	\item Los antecedentes sobre la estimación de la extensión del glaciar Quelcaya son limitados, y no se encontraron datos oficiales de instituciones gubernamentales. Sin embargo, se realizó una comparación de los resultados de superficie glaciar obtenidos en este estudio con los de artículos científicos, especialmente con el trabajo de \parencite{malone2022evolution}, \parencite{taylor2022multi} y \parencite{hanshaw2014glacial}. Los resultados del análisis temporal muestran una alta concordancia con los estudios previos, con estimaciones de superficie glaciar muy cercanas, como se observa en los años 1991, 1995, 1999, 2009, 2015, 2019 y 2020. En este estudio, las estimaciones de superficie glaciar fueron de 50.81, 48.71, 46.26, 41.41, 41.59, 38.42 y 38.99 km², mientras que \parencite{malone2022evolution} reportó valores de 50.89, 48.89, 46.44, 41.69, 41.65, 38.79 y 39.03 km² en los mismos años. Para los años 1991, 1992, 2006, 2016, 2019 . En este estudio, las estimaciones de superficie glaciar furon de 50.81, 49.45, 44.19, 39.25, 38.42 km², mientras que \parencite{taylor2022multi} reportó valores de 50.21, 49.08, 44.79, 40.57, 39.58 km². Para el año 2006 en este estudio, la estimación glaciar fue de 44.19 km², mientras que en \parencite{hanshaw2014glacial} reportó un valor de 44.4 km². Se puede concluir que muchos de los valores obtenidos en este estudio son muy proximos a los antecedentes mencionados, en general las estimaciones de superficie glaciar obtenidas en este estudio muestran que, los valores obtenidos son muy cercanos a los valores reportados por \parencite{malone2022evolution}, \parencite{taylor2022multi} y \parencite{hanshaw2014glacial}, lo que sugiere que los resultados obtenidos mediante aprendizaje profundo son comparables a los obtenidos con métodos tradicionales como el NDSI. 
	
	Es importante destacar que el método tradicional NDSI utilizado en los estudios previos y el método de segmentación semántica basado en deep learning empleado en este estudio son técnicamente diferentes. Sin embargo, esta comparación demuestra que el aprendizaje profundo puede ser una alternativa competitiva y precisa para la estimación de la superficie glaciar, ofreciendo resultados similares a los de métodos más tradicionales.
\end{enumerate}

\singlespacing

\chapter*{Recomendaciones}
\doublespacing
\begin{enumerate}
	\item Para mejorar la precisión y coherencia en los resultados, se recomienda utilizar datos ópticos con mejor resolución, como los del satélite Sentinel, que tiene una resolución de 10 metros por píxel en las bandas de color visible (rojo, verde, azul) y en el infrarrojo cercano, y 20 metros por píxel en las bandas de infrarrojo de onda corta y otras bandas del espectro visible. Alternativamente, se puede utilizar el satélite peruano PerúSAT-1, que ofrece una resolución de 0.7 metros por píxel en su modo pancromático (blanco y negro) y aproximadamente 2 metros por píxel en modo multiespectral (imágenes en color y en diferentes bandas del espectro). Gracias a esta resolución, se pueden capturar detalles de gran precisión en la superficie terrestre, lo que resulta útil para aplicaciones en cartografía, monitoreo de recursos naturales, agricultura, y estudios medioambientales. Sin embargo, estas imágenes no cuentan con datos históricos.
	\item También se recomienda incluir datos de bandas térmicas junto con la temperatura promedio y precipitación de la zona o región, para enriquecer el conjunto de datos. El uso de más variables que afectan a los glaciares contribuirá a precisar los modelos de deep learning, ya que cuanto más datos se utilicen para entrenar el modelo, más coherentes y precisos serán los resultados.
	\item Para futuras investigaciones, se recomienda una combinación de datos ópticos, radares y sensores LIDAR. Esta metodología permitirá construir series temporales mucho más precisas y coherentes para los cambios en los perímetros de los glaciares. Aunque esta información es aún limitada, ya existen antecedentes, como los estudios realizados por \parencite{montoya2024estimation} para los cuerpos glaciares de Suyuparina y Quisoquipina.
\end{enumerate}
\singlespacing