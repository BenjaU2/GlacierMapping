\chapter{Recomendaciones}
\doublespacing
\begin{comment}
	Para mejorar la presición y coherencia en los resultados se recomienda utilizar datos ópticos con mejor resolución como sentinel que tien una resolución 10 metros por píxel en las bandas de color visible (rojo, verde, azul) y infrarrojo cercano y 20 metros por píxel en las bandas de infrarrojo de onda corta y algunas bandas adicionales del espectro visible.
	o incluso nuestro satelite peruano PerúSAT-1 que cuenta con una resolución de 0.7 metros por píxel en su modo pancromático (blanco y negro) y aproximadamente 2 metros por píxel en modo multiespectral (imágenes en color y en diferentes bandas del espectro). Gracias a esta resolución, se podria capturar detalles de gran precisión en la superficie terrestre, siendo útil para aplicaciones en cartografía, monitoreo de recursos naturales, agricultura, y estudios medioambientales, entre otros, sin embargo estas imagenes no cuentas con datos historicos. 
	
	También se recomienda el uso de bandas termicas, junto con los datos de temperatura promedio y presipitación de la zona o región que se podrian incluir al conjunto de datos, el uso de mas variables que afectan a los glaciares ayudaria a precisar el modelo de deep learning, ya que cuanto mas datos se utilicen para entrenar el modelo, mas coherente y presiso seran los resultados .
	
	
	Para futuras investigaciones se recomienda utilizar una combinación de datos ópticos, radares y sensores LIDAR, esto permitira construir series temporales mucho más precisas y coherentes para los cambios de perímetros, aunque la información aplicando esta metodologia es muy poco ya existen antecedentes como las que hizo  \parencite{montoya2024estimation} para los cuerpos glaciares de Suyuparina y Quisoquipina.
	contenidos...
\end{comment}
\begin{enumerate}
	\item Para mejorar la precisión y coherencia en los resultados, se recomienda utilizar datos ópticos con mejor resolución, como los del satélite Sentinel, que tiene una resolución de 10 metros por píxel en las bandas de color visible (rojo, verde, azul) y en el infrarrojo cercano, y 20 metros por píxel en las bandas de infrarrojo de onda corta y otras bandas del espectro visible. Alternativamente, se puede utilizar el satélite peruano PerúSAT-1, que ofrece una resolución de 0.7 metros por píxel en su modo pancromático (blanco y negro) y aproximadamente 2 metros por píxel en modo multiespectral (imágenes en color y en diferentes bandas del espectro). Gracias a esta resolución, se pueden capturar detalles de gran precisión en la superficie terrestre, lo que resulta útil para aplicaciones en cartografía, monitoreo de recursos naturales, agricultura, y estudios medioambientales. Sin embargo, estas imágenes no cuentan con datos históricos.
	\item También se recomienda incluir datos de bandas térmicas junto con la temperatura promedio y precipitación de la zona o región, para enriquecer el conjunto de datos. El uso de más variables que afectan a los glaciares contribuirá a precisar los modelos de deep learning, ya que cuanto más datos se utilicen para entrenar el modelo, más coherentes y precisos serán los resultados.
	\item Para futuras investigaciones, se recomienda una combinación de datos ópticos, radares y sensores LIDAR. Esta metodología permitirá construir series temporales mucho más precisas y coherentes para los cambios en los perímetros de los glaciares. Aunque esta información es aún limitada, ya existen antecedentes, como los estudios realizados por \parencite{montoya2024estimation} para los cuerpos glaciares de Suyuparina y Quisoquipina.
\end{enumerate}
\singlespacing